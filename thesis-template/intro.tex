%#! platex thesis.tex

%======================================================================
\chapter{はじめに}
\label{cha:intro}

%----------------------------------------------------------------------
\section{研究背景}
近年, ロボットの研究が注目されている. 中でも知能ロボットと呼ばれる人間の手足や指などに相当する運動機能のほかに, 視覚, 触覚, 聴覚などの感覚機能, および学習, 連想, 記憶, 推論などの思考機能を備えたロボットの研究がめざましい. 従来のロボットに比べ柔軟に対応することができる知能ロボットはより様々な場で活躍することが期待される. 例えば, 災害現場で人が立ち入るのが困難な場所へ向かい周囲の情報を提供したり, 被災者を発見したりといったことを行う災害救助ロボットや, 掃除や洗濯といったことを行う家庭用マルチサービスロボットなどが考えられる. これらのロボットではコンテキストと呼ばれる周囲の状況や, 内部の状態によって振る舞いを変えることが必要となる. 災害救助ロボットでは, 災害の状況によって移動方法を車輪からプロペラに変えたり, バッテリーの残量に応じて機能を制限したりする必要がある. また, 家庭用マルチサービスロボットでは, 周囲の湿度に合わせて掃除の方法を乾拭きから水拭きに変えたり, 屋内にいるか屋外にいるかで自己位置推定の方法を変更したりする必要がある. このようなコンテキストに応じて振る舞いを変えるようなロボットのことをコンテキストアウェアなロボットとする.\par
現在, ロボットの開発プラットフォームの標準化に対する研究が行われており, 中でもROS(ロボットオペレーティングシステム)と呼ばれるオープンソースのロボットフォフトウェアが注目されている. ROSはメッセージベースのピアツーピア型のロボットミドルウェアであり, ROS上で開発されたソフトウェアモジュールは, 汎用性, 再利用性, 移植性に優れている.\par
コンテキストに依存する振る舞いを扱うための技術としてCOP(コンテキスト指向プログラミング)が提案されている. COPを用いることでコンテキストに依存する振る舞いの変更が可能になる.

%----------------------------------------------------------------------
\section{提案手法}
本論文ではロボットオペレーティングシステムにコンテキスト指向プログラミングを適用したContextROSを提案する.ContextROSでは,コンテキストの変更に応じた振る舞いの変更を可能にする.また,コンテキスト依存な振る舞いをまとめて記述することでコードの再利用性を高めている.



%----------------------------------------------------------------------
\section{論文の構成}
本論文の構成は以下の通りである.第2章ではコンテキストアウェアなロボットの開発に関する技術と既存研究を紹介する.第3章では提案手法についての説明を行う.第4章では提案手法のの評価を行う.最後に第5章でまとめとし,本研究の主たる成果と今後の課題について言及する.






% 以下はRefTeX用
%%% Local Variables:
%%% mode: yatex
%%% TeX-master: "bt_saeki"
%%% End:
