%#! platex thesis.tex

%======================================================================
\chapter{はじめに}
\label{cha:intro}

%----------------------------------------------------------------------
\section{研究背景}
近年,ロボットの研究が注目されている.中でも知能ロボットと呼ばれる人間の手足や指などに相当する運動機能のほかに,視覚,触覚,聴覚などの感覚機能,および学習,連想,記憶,推論,などの思考機能を備えたロボットの研究がめざましい.従来のロボットに比べ柔軟に対応することができる知能ロボットはより様々な場で活躍することが期待される.例えば,災害現場で人が立ち入るのが困難な場所へ向かい周囲の情報を提供したり,被災者を発見したりといったことを行う災害救助ロボットや,掃除や洗濯といったことを行う家庭用マルチサービスロボットなどが考えられる.これらのロボットではコンテキストと呼ばれる周囲の状況や,内部の状態によって振る舞いを変えることが必要となる.災害救助ロボットでは,災害の状況によって移動方法を車輪からプロペラに変えたり,バッテリーの残量に応じて〜したりする必要がある.家庭用マルチサービスロボットでは,周囲の湿度に合わせて掃除の方法を乾拭きから水拭きに変えたり,屋内にいるか屋外にいるかで自己位置推定の方法を変更したりする必要がある.このようなコンテキストに応じて振る舞いを変えるようなロボットのことをコンテキストアウェアなロボットとする.



%----------------------------------------------------------------------
\section{論文の構成}







% 以下はRefTeX用
%%% Local Variables:
%%% mode: yatex
%%% TeX-master: "bt_saeki"
%%% End:
