%#! platex bt_saeki.tex
\documentclass[a4paper,12pt]{jreport}
\usepackage{jgraduate}          % 卒論・修論用スタイル

%% 索引作成
%\usepackage{makeidx}            
% dvipdfmxを使用しない場合はオプションを変更すること
\usepackage[dvipdfmx]{graphicx}
% 数字付きリストでラベルを使う
\usepackage{enumerate}          
% 数学記号など
\usepackage{amsmath}
\usepackage{amssymb}
\usepackage{amsthm}
% URLをいい感じにする
\usepackage{url}
\usepackage{cite}

%------------------------------
% 余白設定
%------------------------------
\usepackage[left=27mm,right=27mm,top=45mm,bottom=45mm,%
 headheight=5mm,headsep=10mm,%
 footskip=12mm%
 ]{geometry}

%------------------------------
% hyperref
%  日本語の文字コード設定が分からない人はコメントアウトすること.
%------------------------------
% PDF化したときにしおりが作成され,図表へのジャンプも可能となる.
\usepackage{atbegshi}
% Mac/Linuxの場合
%   ※Ubuntuの場合はEUC-UCS2を自分で入れないとダメ.
\AtBeginShipoutFirst{\special{pdf:tounicode EUC-UCS2}}
%% Winの場合
%\AtBeginShipoutFirst{\special{pdf:tounicode 90ms-RKSJ-UCS2}}
% 以下は共通
\usepackage[dvipdfm,a4paper,bookmarks,bookmarksnumbered,%
 bookmarksopen=false,pdfstartview={FitH},%
 bookmarkstype=toc,%
 setpagesize=false,%
 pdfauthor={佐伯 優太},%
% setpagesize=false,% PDFのサイズがおかしい場合はこれを有効化
 pdftitle={俺の研究がこんなにすごいわけがない!}]{hyperref}

%----------------------------------------------------------------------
% 設定
%----------------------------------------------------------------------
% 目次の深さはsubsubsectionまで
\setcounter{tocdepth}{3}

% 基準となる図の幅
\newlength\figurewidth
\setlength{\figurewidth}{0.8\textwidth}
% 縦に並べた図の間の基準となるスペース
\newlength\figuresep
\setlength{\figuresep}{0.8\floatsep}

%----------------------------------------------------------------------
% 文書基本情報
%----------------------------------------------------------------------
% タイトル
\title{俺の研究が\\こんなに\\すごいわけがない!}

% 著者
\author{佐伯 優太}

% 所属
\university{九州大学}
\department{工学部}
\major{電気情報工学科}
%\university{九州大学大学院}
%\department{システム情報科学府}
%\major{情報知能工学専攻}

% 提出日(月までを書く)
\date{平成29年2月}

%% 書いている途中では以下のようにしておくと一部だけをタイプセットできる
%\includeonly{intro}

%======================================================================
% テキスト開始
%======================================================================
\begin{document}
% 表紙
\maketitle
% 表紙はページ番号を出力しない
\thispagestyle{empty}

%----------------------------------------------------------------------
% 概要
%----------------------------------------------------------------------
\begin{abstract}
%これは修論・卒論のテンプレートである.
コンテキストアウェアなロボットの開発のためには、コンテキストの変化に対して振る舞いを変更するメカニズムを提供する必要がある.本稿ではロボットオペレーティングシステム(ROS)上でコンテキストに依存する振る舞いを変更させる仕組みContextROSを提案する.ContextROSはコンテキストに依存する振る舞いをレイヤにまとめて記述することができる.また、文脈依存の振る舞いをコンパイル時に静的に解釈することで振る舞いの変更を行うことができる.
\end{abstract}

%----------------------------------------------------------------------
% 目次のページ番号は1から
\setcounter{page}{0}
% 目次
\tableofcontents

% 本文のページ番号はアラビア数字
\pagenumbering{arabic}
 
%======================================================================
% 本文ここから
%======================================================================

% 章ごとのファイルを読み込む
% YaTeXなら各ファイルを読み込む行でC-c gでジャンプできる
%#! platex thesis.tex

%======================================================================
\chapter{はじめに}
\label{cha:intro}

%----------------------------------------------------------------------
\section{研究背景}
近年, ロボットの研究が注目されている. 中でも知能ロボットと呼ばれる人間の手足や指などに相当する運動機能のほかに, 視覚, 触覚, 聴覚などの感覚機能, および学習, 連想, 記憶, 推論などの思考機能を備えたロボットの研究がめざましい. 従来のロボットに比べ柔軟に対応することができる知能ロボットはより様々な場で活躍することが期待される. 例えば, 災害現場で人が立ち入るのが困難な場所へ向かい周囲の情報を提供したり, 被災者を発見したりといったことを行う災害救助ロボットや, 掃除や洗濯といったことを行う家庭用マルチサービスロボットなどが考えられる. これらのロボットではコンテキストと呼ばれる周囲の状況や, 内部の状態によって振る舞いを変えることが必要となる. 災害救助ロボットでは, 災害の状況によって移動方法を車輪からプロペラに変えたり, バッテリーの残量に応じて機能を制限したりする必要がある. また, 家庭用マルチサービスロボットでは, 周囲の湿度に合わせて掃除の方法を乾拭きから水拭きに変えたり, 屋内にいるか屋外にいるかで自己位置推定の方法を変更したりする必要がある. このようなコンテキストに応じて振る舞いを変えるようなロボットのことをコンテキストアウェアなロボットとする.\par
現在, ロボットの開発プラットフォームの標準化に対する研究が行われており, 中でもROS(ロボットオペレーティングシステム)と呼ばれるオープンソースのロボットソフトウェアが注目されている. ROSはメッセージベースのピアツーピア型のロボットミドルウェアであり, ROS上で開発されたソフトウェアモジュールは, 汎用性, 再利用性, 移植性に優れている.\par
コンテキストに依存する振る舞いを扱うための技術としてCOP(コンテキスト指向プログラミング)が提案されている. COPを用いることでコンテキストに依存する振る舞いの変更が可能になる.

%----------------------------------------------------------------------
\section{提案手法}
本論文ではロボットオペレーティングシステムにコンテキスト指向プログラミングを適用したContextROSを提案する.ContextROSでは,コンテキストの変更に応じた振る舞いの変更を可能にする.また,コンテキスト依存な振る舞いをまとめて記述することでコードの再利用性を高めている.



%----------------------------------------------------------------------
\section{論文の構成}
本論文の構成は以下の通りである.第2章ではコンテキストアウェアなロボットの開発に関する技術と既存研究を紹介する.第3章では提案手法についての説明を行う.第4章では提案手法のの評価を行う.最後に第5章でまとめとし,本研究の主たる成果と今後の課題について言及する.






% 以下はRefTeX用
%%% Local Variables:
%%% mode: yatex
%%% TeX-master: "bt_saeki"
%%% End:

%#! platex bt_saeki.tex

%======================================================================
\chapter{要素技術}
\label{cha:element_technology}

%----------------------------------------------------------------------
\section{コンテキスト指向プログラミング}
%--------------------------------------------------
\subsection{概要}

%----------------------------------------------------------------------
\section{ロボットオペレーティングシステム}
%--------------------------------------------------
\subsection{概要}


% 以下はRefTeX用
%%% Local Variables:
%%% mode: yatex
%%% TeX-master: "bt_saeki"
%%% End:
%#! platex bt_saeki.tex

%======================================================================
\chapter{関連技術}
\label{cha:related_work}
本章では本研究に用いる要素技術についての説明と関連研究の紹介を行う.
はじめに, コンテキスト指向プログラミングについて説明を行う. コンテキスト指向プログラミングはコンテキストに依存する振る舞いの変更を可能にする.
次に, ロボットオペレーティングシステムについて説明する.
その後, 関連研究について説明を行う.


%----------------------------------------------------------------------
\section{コンテキスト指向プログラミング}
コンテキストアウェアな振る舞いを記述する手法として, COP(コンテキスト指向プログラミング)が提案されている[Context-orientedPrograming]. これは, コンテキストに依存する振る舞いをレイヤと呼ばれるものにモジュール化し, コンテキストの変更に合わせてレイヤをアクティベート, 非アクティベートすることで振る舞いを変更する[]. COPの記述の例を図に示す. COPに関する研究は数多く存在し, COP言語には様々なものが存在する. 
%--------------------------------------------------
%\subsection{概要}

%----------------------------------------------------------------------
\section{ロボットオペレーティングシステム}

%--------------------------------------------------
%\subsection{概要}


%----------------------------------------------------------------------
\section{関連研究}




% 以下はRefTeX用
%%% Local Variables:
%%% mode: yatex
%%% TeX-master: "bt_saeki"
%%% End:
%#! platex bt_saeki.tex

%======================================================================
\chapter{提案手法}
\label{cha:proposed_method}
第2章で述べたCOP, ROS, 2つの技術を組み合わせることでより汎用的なコンテキストアウェアなロボットの開発を可能とするContextROSを提案する. ContextROSはコンテキスト依存な振る舞いのモジュール化と, レイヤのアクティベーション, 非アクティベーションによる振る舞いの変更を行う.
%----------------------------------------------------------------------
\section{概要}
本節では, ContextROSの概要について述べる. ContextROSはROS上でコンテキストに応じた振る舞いの変更を容易にすることを目的とする. コンテキスト依存な振る舞いをCOPの力を借りることで容易に扱えるようにする. ContextROSは2つの要素によってCOPを実現している. 1つはレイヤ記述によるコンテキスト依存な振る舞いのモジュール化. もう1つは, ROSの通信を用いたレイヤアクティベーションによる振る舞いの変更である.ContextROSの全体図を図に示す. \par
レイヤ記述を解釈しレイヤごとに振る舞いを変更する関数を生成. \par
レイヤコントローラがコンテキスト情報を受け取りアクティベートするレイヤを変更する. \par
アクティベートするレイヤをROSのTopic通信を用いて配布する. \par
アクティベート中のレイヤ情報を引数に生成された関数を呼び出すことでレイヤごとの振る舞いの変更を実現. \par

%----------------------------------------------------------------------
\section{レイヤ記述}
本節では, レイヤ記述について説明する. (レイヤ記述とは何か?)
はじめに, レイヤ記述の構成について述べたのち, その記述の解釈について述べる.

%--------------------------------------------------
\subsection{構成}

(BNFを使うべき?)\par
レイヤの記述はLayer"レイヤ名"[関数定義]の形で定義される振る舞い定義部とそれ以外の部分からなる.


%--------------------------------------------------
\subsection{解釈}
\begin{itemize}
 \item レイヤ内に書かれたそれぞれの関数の名前にレイヤ名をつけ新しく定義する.
 \item それぞれのレイヤ内に書かれた共通の名前の関数を宣言し, レイヤ番号でif文を生成する
 \item レイヤの記述以外の部分と新たに生成した関数の定義を生成する.
\end{itemize}

%----------------------------------------------------------------------
\section{レイヤアクティベーション}

\begin{enumerate}
 \item コンテキストの変更に伴うアクティブレイヤの変更依頼
 \item アクティブレイヤの変更依頼で指定のあったレイヤ番号をトピック通信で配布
 \item アクティブなレイヤ番号を参照し振る舞いを変更する
\end{enumerate}

%----------------------------------------------------------------------


%----------------------------------------------------------------------


%----------------------------------------------------------------------


%--------------------------------------------------


%--------------------------------------------------



% 以下はRefTeX用
%%% Local Variables:
%%% mode: yatex
%%% TeX-master: "bt_saeki"
%%% End:
%#! platex bt_saeki.tex

%======================================================================
\chapter{評価}
\label{cha:evaluation}

%----------------------------------------------------------------------


%----------------------------------------------------------------------


%----------------------------------------------------------------------


%----------------------------------------------------------------------


%----------------------------------------------------------------------


%----------------------------------------------------------------------

%--------------------------------------------------


%--------------------------------------------------


%--------------------------------------------------


%--------------------------------------------------


%--------------------------------------------------


%--------------------------------------------------



% 以下はRefTeX用
%%% Local Variables:
%%% mode: yatex
%%% TeX-master: "bt_saeki"
%%% End:
%#! platex thesis.tex

%======================================================================
\chapter{おわりに}
\label{cha:conclu}

%----------------------------------------------------------------------
\section{本研究の主たる成果}
\label{sec:main-result}

何もねぇ...

本当にねぇ...

%----------------------------------------------------------------------
\section{今後の課題}
\label{sec:future}

課題なんてあろうか.
いや,あるわけがない.
全てが終わったのだから.


% 以下はRefTeX用
%%% Local Variables:
%%% mode: yatex
%%% TeX-master: "thesis"
%%% End:



% 謝辞
%#! platex thesis.tex
%======================================================================
% 謝辞
%======================================================================
\acknowledgment

あり〜がと〜\\
さよ〜なら〜\\
きょ〜しつ〜\\
しかられたこと〜さえ〜\\
あ〜たた〜か〜い〜

\newpage

謝辞が2ページになったときのテスト.


%----------------------------------------------------------------------
% 参考文献
%----------------------------------------------------------------------
\bibliographystyle{sieicej}
\bibliography{bib/IEEEfull,bib/mystr_IEEEfull,bib/my,bib/pub}
% 書き終えたら,↑の2行をコメントアウトして,BibTeXが生成したthesis.bbl
% をref.texという名前に変更して
% ↓を有効化するとよい.必要があれば手動で修正する.
%\include{ref}

%----------------------------------------------------------------------
% 発表文献
%----------------------------------------------------------------------
%#!platex thesis.tex
%======================================================================
% Publications
%======================================================================
\publications

% 数が少なければ項目に分ける必要はないです.
% section*の区切りを削除して下さい.

%----------------------------------------------------------------------
\section*{学術雑誌等(査読あり) }
\begin{publication}{99}
\bibitem{mlab/ishida11:wake-up_ieice_trans}
石田繁巳,瀧口貴啓,猿渡俊介,南\hskip1zw正輝,森川博之,
``ブルームフィルタを用いたウェイクアップ型通信システム,\<''
電子情報通信学会論文誌B: 通信,
vol.J94-B,no.10,pp.1397--1407,Oct.\ 2011.

\end{publication}

%----------------------------------------------------------------------
%\section*{学術雑誌等または商業誌における解説,総説}
%\begin{publication}{P99}
%\end{publication}

%----------------------------------------------------------------------
\section*{国際会議における発表}
\subsection*{口頭発表(査読あり)}
\begin{publication}{99}
\bibitem{mlab/takiguchi09:wakeup_greencomm}
T. Takiguchi, S. Saruwatari, T. Morito, S. Ishida, M. Minami, and H. Morikawa,
``A novel wireless wake-up mechanism for energy-efficient ubiquitous
  networks,''
Proceedings of the {IEEE} Workshop on Green Communications (GreenComm),
pp.1--5, June 2009.

\bibitem{mlab/ishida10:apsitt}
S. Ishida, T. Takiguchi, S. Saruwatari, M. Minami, and H. Morikawa,
``Evaluation of a wake-up wireless module with bloom-filter-based {ID}
  matching,''
Proceedings of Asia-Pacific Symposium on Information and Telecommunication
  Technologies (APSITT),
pp.1--6, June 2010.

\end{publication}

%--------------------------------------------------
\subsection*{ポスター,デモ発表(査読あり)}
\begin{publication}{99}
\bibitem{mlab/ishida06:percom_demo}
S. Ishida, M. Minami, Y. Nishizawa, T. Morito, Y. Moriyama, H. Morikawa, and T.
  Aoyama,
``Three devices for tackling practical problems in pervasive computing,''
IEEE International Conference on Pervasive Computing and Communications
  (PerCom), Demo,
p.1,
D8, March 2006.

\bibitem{mlab/ishida10:iot}
S. Ishida, T. Takiguchi, S. Saruwatari, M. Minami, and H. Morikawa,
``Implementation of bloom-filter-based {ID} matching for wake-up wireless
  communication,''
Internet of Things 2010 Conference (IoT 2010), poster, Dec.\ 2010.

\end{publication}

%----------------------------------------------------------------------
\section*{研究会}
\begin{publication}{99}
\bibitem{mlab/Ishida08:wakeup_in}
石田繁巳,鈴木\hskip1zw誠,森戸\hskip1zw貴,森川博之,
``低受信待機電力無線通信のための多段ウェイクアップ機構,\<''
電子情報通信学会技術報告,
pp.355--360,
情報ネットワーク研究会(IN2007-218),March 2008.

\bibitem{mlab/takiguchi10:bloom_rcs}
瀧口貴啓,石田繁巳,猿渡俊介,南\hskip1zw正輝,森川博之,
``ブルームフィルタを用いたウェイクアップ型無線通信システムの消費電力評価,\<''
電子情報通信学会技術報告,
pp.269--274,
無線通信システム研究会(RCS2009-254),Jan.\ 2010.

\bibitem{mlab/takiguchi11:wakeup_in}
瀧口貴啓,石田繁巳,岸\hskip1zw孝彦,丹羽栄二,見並一明,猿渡俊介,森川博之,
``ウェイクアップ型無線通信におけるビット不一致許容{ID}マッチング,\<''
電子情報通信学会技術報告,
pp.193--198,
情報ネットワーク研究会(IN2010-176),March 2011.

\end{publication}

%----------------------------------------------------------------------
\section*{全国大会}
\begin{publication}{99}
\bibitem{mlab/matsui07:wake-up_st}
松井壮介,石田繁巳,鈴木\hskip1zw誠,猿渡俊介,森川博之,
``実験的アプローチによるシングルホップ通信とマルチホップ通信の消費電力の比較,%
\<''
電子情報通信学会総合大会,
p.1,
A-21-22,March 2007.

\bibitem{mlab/ishida07:wake-up_st}
石田繁巳,猿渡俊介,鈴木\hskip1zw誠,森川博之,
``サービス発見のためのゼロ受信待機電力無線システムの設計,\<''
電子情報通信学会総合大会,
p.1,
B-7-202,March 2007.

\bibitem{mlab/ishida08:wake-up_st}
石田繁巳,鈴木\hskip1zw誠,森戸\hskip1zw貴,森川博之,
``低受信待機電力無線通信のための階層型ウェイクアップ機構,\<''
電子情報通信学会総合大会,
p.1,
B-5-112,March 2008.

\bibitem{mlab/ishida10:wake-up_imple}
石田繁巳,瀧口貴啓,猿渡俊介,南\hskip1zw正輝,森川博之,
``ウェイクアップ型無線通信のためのグループ指定可能{ID}マッチング機構の実装,\<%
''
電子情報通信学会ソサイエティ大会,
p.1,
B-5-140,Sept.\ 2010.

\bibitem{mlab/ishida11:wake-up_sogo}
石田繁巳,瀧口貴啓,猿渡俊介,森川博之,
``ブルームフィルタを用いたウェイクアップ型無線通信システムにおける{ID}長の影響%
,\<''
電子情報通信学会総合大会,
p.1,
B-5-146,March 2011.

\bibitem{mlab/takiguchi11:wake-up_sogo}
瀧口貴啓,石田繁巳,岸\hskip1zw孝彦,丹羽栄二,見並一明,猿渡俊介,森川博之,
``車両内ウェイクアップ型無線通信における数個のビット不一致許容{ID}マッチング,%
\<''
電子情報通信学会総合大会,
p.1,
B-5-145,March 2011.

\bibitem{mlab/okamura11:sogo}
岡村悠貴,鈴木\hskip1zw誠,石田繁巳,今泉英明,関谷勇司,森川博之,
``非同期光パケットリングにおける高帯域利用効率パケット選択方式,\<''
電子情報通信学会総合大会,
p.1,
B-10-96,March 2011.

\bibitem{mlab/ishida11:mixer_sotai}
石田繁巳,鈴木\hskip1zw誠,森川博之,
``サブスレッショルド特性を利用するウェイクアップ受信機用ミキサの初期的検討,\<%
''
電子情報通信学会ソサイエティ大会,
p.1,
C-12-16,Sept.\ 2011.

\bibitem{mlab/kim11:spectrum_sotai}
金\hskip1zw昊俊,長縄潤一,石田繁巳,鈴木\hskip1zw誠,森川博之,
``可変{RBW}を用いた周波数占有率の測定精度の初期的評価,\<''
電子情報通信学会ソサイエティ大会,
p.1,
B-17-8,Sept.\ 2011.

\bibitem{mlab/nakamura11:co2_visual_sotai}
中村元紀,中村隆幸,荒川\hskip1zw豊,東島由佳,柏木啓一郎,森\hskip1zw皓平,松%
村\hskip1zw一,石田繁巳,猿渡俊介,翁長\hskip1zw久,森川博之,
``{uTupleSpace}を利用した$\mathrm{CO_2}$排出量可視化の実証実験,\<''
電子情報通信学会ソサイエティ大会,
p.1,
B-19-21,Sept.\ 2011.

\bibitem{mlab/nakajima12:spfc_sogo}
中嶋毅彰,米川\hskip1zw慧,石田繁巳,鈴木\hskip1zw誠,森川博之,
``多様なサービス電力の発見・割当て・制御機構,\<''
電子情報通信学会総合大会,
p.1,
BS-4-2,March 2012.

\end{publication}

%----------------------------------------------------------------------
\section*{その他の学会等}
\begin{publication}{99}
\bibitem{mlab/ishida09:inria}
S. Ishida,
``Wake-up wireless communication system for energy-efficient ubiquitous
  network,''
INRIA-TODAI Workshop (GCOE-INRIA Workshop), oral presentation, Dec.\ 2009.

\bibitem{mlab/ishida10:hakone_mct}
S. Ishida,
``Design of a zero-power-listening wireless system for service discovery,''
1st International Workshop on Microwatt Communication Technology, Jan.\ 2010.

\bibitem{mlab/ishida10:google_workshop}
S. Ishida,
``Wake-up wireless communication system,''
Tech Talks and Mix at Google Tokyo, oral presentation, Dec.\ 2010.

\bibitem{mlab/kak-t10:health_poster}
角田\hskip1zw仁,中嶋毅彰,石田繁巳,猿渡俊介,森川博之,
``社会実装に向けたヘルスケア情報共有基盤,\<''
第5回人間情報学会講演会,ポスター,Dec.\ 2010.

\end{publication}


\end{document}