%#! platex bt_saeki.tex

%======================================================================
\chapter{提案手法}
\label{cha:proposed_method}
第2章で述べたCOP, ROS, 2つの技術を組み合わせることでより汎用的なコンテキストアウェアなロボットの開発を可能とするContextROSを提案する. ContextROSはコンテキスト依存な振る舞いのモジュール化と, レイヤのアクティベーション, 非アクティベーションによる振る舞いの変更を行う.
%----------------------------------------------------------------------
\section{概要}
本節では, ContextROSの概要について述べる. ContextROSはROS上でコンテキストに応じた振る舞いの変更を容易にすることを目的とする. コンテキスト依存な振る舞いをCOPの力を借りることで容易に扱えるようにする. ContextROSは2つの要素によってCOPを実現している. 1つはレイヤ記述によるコンテキスト依存な振る舞いのモジュール化. もう1つは, ROSの通信を用いたレイヤアクティベーションによる振る舞いの変更である.ContextROSの全体図を図に示す. \par
レイヤ記述を解釈しレイヤごとに振る舞いを変更する関数を生成. \par
レイヤコントローラがコンテキスト情報を受け取りアクティベートするレイヤを変更する. \par
アクティベートするレイヤをROSのTopic通信を用いて配布する. \par
アクティベート中のレイヤ情報を引数に生成された関数を呼び出すことでレイヤごとの振る舞いの変更を実現. \par

%----------------------------------------------------------------------
\section{レイヤ記述}
本節では, レイヤ記述について説明する. (レイヤ記述とは何か?)
はじめに, レイヤ記述の構成について述べたのち, その記述の解釈について述べる.

%--------------------------------------------------
\subsection{構成}

(BNFを使うべき?)\par
レイヤの記述はLayer"レイヤ名"[関数定義]の形で定義される振る舞い定義部とそれ以外の部分からなる.


%--------------------------------------------------
\subsection{解釈}
\begin{itemize}
 \item レイヤ内に書かれたそれぞれの関数の名前にレイヤ名をつけ新しく定義する.
 \item それぞれのレイヤ内に書かれた共通の名前の関数を宣言し, レイヤ番号でif文を生成する
 \item レイヤの記述以外の部分と新たに生成した関数の定義を生成する.
\end{itemize}

%----------------------------------------------------------------------
\section{レイヤアクティベーション}

\begin{enumerate}
 \item コンテキストの変更に伴うアクティブレイヤの変更依頼
 \item アクティブレイヤの変更依頼で指定のあったレイヤ番号をトピック通信で配布
 \item アクティブなレイヤ番号を参照し振る舞いを変更する
\end{enumerate}

%----------------------------------------------------------------------


%----------------------------------------------------------------------


%----------------------------------------------------------------------


%--------------------------------------------------


%--------------------------------------------------



% 以下はRefTeX用
%%% Local Variables:
%%% mode: yatex
%%% TeX-master: "bt_saeki"
%%% End: