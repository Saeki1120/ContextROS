%#! platex bt_saeki.tex

%======================================================================
\chapter{提案手法}
\label{cha:proposed_method}
第2章で述べたCOP, ROS, 2つの技術を組み合わせることでより汎用的なコンテキストアウェアなロボットの開発を可能とするContextROSを提案する. ContextROSはコンテキスト依存な振る舞いのモジュール化と, レイヤのアクティベーション, 非アクティベーションによる振る舞いの変更を行う.
%----------------------------------------------------------------------
\section{概要}
本節では, ContextROSの概要について述べる. ContextROSはROS上でコンテキストに応じた振る舞いの変更を容易にすることを目的とする. コンテキスト依存な振る舞いをCOPの力を借りることで容易に扱えるようにする. ContextROSは2つの要素によってCOPを実現している. 1つはレイヤ記述によるコンテキスト依存な振る舞いのモジュール化. もう1つは, ROSの通信を用いたレイヤアクティベーションによる振る舞いの変更である. 

%----------------------------------------------------------------------


%----------------------------------------------------------------------


%----------------------------------------------------------------------


%----------------------------------------------------------------------


%----------------------------------------------------------------------

%--------------------------------------------------


%--------------------------------------------------


%--------------------------------------------------


%--------------------------------------------------


%--------------------------------------------------


%--------------------------------------------------



% 以下はRefTeX用
%%% Local Variables:
%%% mode: yatex
%%% TeX-master: "bt_saeki"
%%% End: