%#! platex bt_saeki.tex

%======================================================================
\chapter{関連技術}
\label{cha:related_work}
本章では本研究に用いる要素技術についての説明と関連研究の紹介を行う.
はじめに, コンテキスト指向プログラミングについて説明を行う. コンテキスト指向プログラミングはコンテキストに依存する振る舞いの変更を可能にする.
次に, ロボットオペレーティングシステムについて説明する.
その後, 関連研究について説明を行う.


%----------------------------------------------------------------------
\section{コンテキスト指向プログラミング}
コンテキストアウェアな振る舞いを記述する手法として, COP(コンテキスト指向プログラミング)が提案されている[Context-orientedPrograming]. これは, コンテキストに依存する振る舞いをレイヤと呼ばれるものにモジュール化し, コンテキストの変更に合わせてレイヤをアクティベート, 非アクティベートすることで振る舞いを変更する[]. COPの記述の例を図に示す. COPに関する研究は数多く存在し, COP言語には様々なものが存在する. 

\begin{itemize}
 \item コンテキスト依存な振る舞いをレイヤと呼ばれる言語要素でモジュール化
 \item (COPの記述例の図を用いて説明)
 \item コンテキストに応じてレイヤアクティベーションと呼ばれるレイヤの切り替えを行い振る舞いを変更
 \item (発表に使ったような図を用いて説明)
\end{itemize}



%--------------------------------------------------
%\subsection{概要}

%----------------------------------------------------------------------
\section{ロボットオペレーティングシステム}

\begin{itemize}
 \item ロボット用のミドルウェア
 \item 分散処理システム
 \item 汎用性のあるロボット機能をノードと呼ばれる機能モジュールに表現
 \item ノードの組み合わせにより複雑な制御機能を分散的に表現
 \item ノードモジュールをパッケージにまとめて機能モジュールの共有と配布を行う
\end{itemize}

%--------------------------------------------------
\subsection{特徴}

\begin{itemize}
 \item ピアツーピア設計方式
 \item オープンソースのプラットフォーム
 \item 補助ツールが豊富
 \item 複数の言語に対応
\end{itemize}

%--------------------------------------------------
\subsection{機能}

\begin{itemize}
 \item プロセス間のメッセージパッシング
 \item トピック通信
 \item サービス通信
\end{itemize}

%----------------------------------------------------------------------
\section{関連研究}

\begin{itemize}
 \item COPのサーベイ
 \item ロボットミドルウェアのサーベイ
\end{itemize}


% 以下はRefTeX用
%%% Local Variables:
%%% mode: yatex
%%% TeX-master: "bt_saeki"
%%% End: