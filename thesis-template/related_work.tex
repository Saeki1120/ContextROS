%#! platex bt_saeki.tex

%======================================================================
\chapter{関連技術}
\label{cha:related_work}
本章では本研究に用いる要素技術についての説明と関連研究の紹介を行う.
はじめに, コンテキスト指向プログラミングについて説明を行う. コンテキスト指向プログラミングはコンテキストに依存する振る舞いの変更を可能にする.
次に, ロボットオペレーティングシステムについて説明する.
その後, 関連研究について説明を行う.


%----------------------------------------------------------------------
\section{コンテキスト指向プログラミング}
コンテキストアウェアな振る舞いを記述する手法として, COP(コンテキスト指向プログラミング)が提案されている[Context-orientedPrograming]. これは, コンテキストに依存する振る舞いをレイヤと呼ばれるものにモジュール化し, コンテキストの変更に合わせてレイヤをアクティベート, 非アクティベートすることで振る舞いを変更する[]. COPの記述の例を図に示す. COPに関する研究は数多く存在し, COP言語には様々なものが存在する. 
%--------------------------------------------------
%\subsection{概要}

%----------------------------------------------------------------------
\section{ロボットオペレーティングシステム}

%--------------------------------------------------
%\subsection{概要}


%----------------------------------------------------------------------
\section{関連研究}




% 以下はRefTeX用
%%% Local Variables:
%%% mode: yatex
%%% TeX-master: "bt_saeki"
%%% End: